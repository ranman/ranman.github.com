%
% portfolio.tex
%
% Portfolio Template
% Western Carolina University
%
% This is the main document for your portfolio.  It specifics the values of
% some variables that will be used within your document.  You can also add
% appendices at the end.
%

%
% The document guidelines say the font can be between 10pt and 12pt.
% Specify whatever you want it to be here.
%
\documentclass[11pt]{WcuPortfolio}

%
% Use any additional packages you might need
%
\usepackage{comment}
\usepackage{mips}


%
% Make the document your own -- fill in these values to reflect the type of
% document you are writing.
%
\department{Department of Mathematics and Computer Science} % Don't change this
\documentType{Portfolio}                                    % Don't change this
\major{Computer Science}                                    % Don't change this
\degree{Bachelor of Science}                                % Don't change this

\author{Joseph Randall Hunt}    % Who are you?
\title{Computer Science Portfolio}      % What's the name of your document?
\graduationMonth{May}       % In what month do you expect to graduate?
\graduationYear{2013}       % In what year do you expect to graduate?

% Who taught each of the classes that you're using for your SLOs?
% If you haven't taken the corresponding course, just comment out
% the corresponding \xxxInstructor below.  You should only include
% the ones that you've had to date.
%\capstoneInstructor{Instructor Name}
%\communicationInstructor{Instructor Name}
%\ethicsInstructor{Dr. Kreahling}
%\softwareEngInstructor{Instructor Name}
%\teamworkInstructor{Instructor Name}
\systemsInstructor{Dr. Kreahling}
%\algorithmsInstructor{Instructor Name}

%
% PDF Setup -- You don't need to touch this stuff.
%
\hypersetup{
    colorlinks,
    linkcolor={blue},
    citecolor={blue},
    filecolor={black},
    urlcolor={black},
    pdftitle={\theTitle},
    pdfauthor={\theAuthor},
    pdfsubject={\theDocumentType},
    pdfkeywords={Western Carolina University, \theDepartment, \theDocumentType, \theMajor, \theDegree},
    pdfstartpage={1},
}

%
% User-specified command definitions/redefinitions.  You can define your own
% commands to make life easier.  Here's an example of how you could create a
% new command called '\cplusplus' that will produce a fancy-looking C++.
%
\newcommand{\cplusplus}{{\rm C\raise.5ex\hbox{\small ++}}}


\begin{document}
%  ============================================================================
    \frontmatter % Begin front matter (pages are numbered with Roman numerals)
%  ============================================================================

    \addtotoc{Title Page}{\maketitle}            % Generate the title page
    \doublespacing                               % Text should be double spaced
    \setcounter{page}{2}                         % Abstract begins on page 2
    \addtotoc{Abstract}{\chapter*{Abstract}\label{abstract}
This portfolio is a collection of the work I've done in Computer Science while attending WCU. I'm writing this so that prospective employers might see examples of my work and have a good understanding of what I'm capable of. Contained within this profile are examples of my work from CS 350 and CS 370. To date I have done several low level assignments for the Systems SLO. At WCU I've learned several things so far including: the importance of well engineered software; the ethics behind programming; the privacy laws and intellectual property laws programmers should be aware of; what is essential for system programming; how low level systems work; and finally the strengths of linux tools and how to use them.
}    % Generate the abstract
    \addtotoc{Consent Form}{\chapter*{Consent Form}\label{consentForm}
I, {\theAuthor}, hereby give the Faculty of the {\theMajor} Program at Western
Carolina University permission to maintain, indefinitely, a copy of any student
portfolio that I develop as part of my course work as an undergraduate in that
program and to use those copies of my student portfolios for program assessment.
\vspace{2cm}

\noindent \begin{tabular}{p{3in}p{1in}p{2in}}
\underline{\hspace{3in}} & & \underline{\hspace{2in}}\\
Signature                & & Date
\end{tabular}
} % Generate the consent form

    \singlespacing                               % Single space the lists
    \tableofcontents \clearpage                  % Generate the ToC

    % If you don't have any tables, comment out the following line
    %\addtotoc{List of Tables}{\listoftables}     % Generate the List of Tables

    % If you don't have any figures, comment out the following line
    %\addtotoc{List of Figures}{\listoffigures}   % Generate the List of Figures

    % Generate the List of Listings
    \addtotoc{List of Listings}{\lstlistoflistings}


%  ===========================================================================
    \mainmatter % Begin main matter (pages are numbered with Arabic numerals)
%  ===========================================================================
    \doublespacing % Text should be double spaced

    %
    % Here we have each chapter in a separate file.  Name these as you choose,
    % and include them in the order you want them to appear.  Be sure to use
    % the \inputfile command.
    %
    \inputfile{introduction.tex}
    \inputfile{studentLearningObjectives.tex}
    \inputfile{conclusions.tex}

    %
    % The appendices are optional.
    % If you do not wish to include an appendix, comment out these lines.
    %
    \begin{appendices}
            \begin{subappendices}
                \inputfile{appendices/appendixA.tex}
                \inputfile{appendices/appendixB.tex}
                \inputfile{appendices/appendixC.tex}
            \end{subappendices}
        \end{appendices}
    
    \singlespacing                             % Single space the Bibliography

    \bibliographystyle{plain}
    \addtotoc{Bibliography}{\bibliography{bibliography}}
\end{document}
